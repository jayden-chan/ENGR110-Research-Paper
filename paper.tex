% Notes:
% - Remove all instances of \blindtext
% - Remove all instances of [placeholder]
% - Check that all quotes and mentions of authors are cited
% - Don't throw uncontrollably

\documentclass[letterpaper]{article}
\usepackage{blindtext}
\usepackage{fancyhdr}
\usepackage{indentfirst}

\setlength{\parindent}{5ex}

\usepackage[
    backend=biber,
    style=ieee,
    url=true,
    citestyle=numeric-comp
]{biblatex}

\addbibresource{references.bib}

\pagestyle{fancy}
\fancypagestyle{customPage}{
    \fancyhf{}
    \renewcommand{\headrulewidth}{0pt}
    \fancyhead[R]{\thepage}
    \fancyhead[C]{\textit{Machine Learning, The Internet of Things, and Sustainable Design}}
}

\begin{document}
\begin{titlepage}
    \centering
    {\scshape\huge UNIVERSITY OF VICTORIA \par}
    \vspace{1cm}
    {\scshape\LARGE Faculty of Engineering \par\Large ENGR 110 A04\par}
    \vspace{1cm}
    {\scshape\Large Research Paper \par}
    \vspace{1.5cm}
    {\huge\bfseries \emph{Machine Learning, The Internet of Things, and Sustainable Design} \par}
    \vspace{2cm}
    {\Large\itshape Jayden Chan \par}
    \vfill
    {\large November 30, 2017\par}
\end{titlepage}

\pagestyle{customPage}

\section{Introduction}
The recent surge in computational power and data set availability has led to a significant increase in the use of artificial intelligence and machine learning (ML) \cite{chan17}. In addition to this, advances in microelectronics and network infrastructure have enabled hardware manufacturers to create internet-connected devices for use in homes and businesses. This technology is commonly referred to as the Internet of Things (IoT). The combination of these two systems can aid significantly in the development of sustainable homes and businesses. Network-enabled cyber-physical devices capable of gathering data from their surroundings can provide optimization algorithms with the data they need to improve automation, energy use efficiency, and [placeholder]. However, integrating these technologies is not without its challenges; the most prominent which involve ethical concerns or technical barriers \cite{perisic16, mccalman17, vlacheas13}. Despite these challenges, the strengths of ML and the IoT strongly outweigh their shortcomings, making them a viable future technology for designing sustainable homes and businesses.

\section{The Technologies}
Before discussing the viability or effectiveness of machine learning and the IoT, it is important to understand how they work.
\subsection{Machine Learning}
While the concept of artificial intelligence predates the common era \cite{mccorduck04}, the term ``machine learning'' was not coined until 1959 by Arthur Samuel \cite{samuel59}. The basic principle behind machine learning is to allow computers to complete tasks without being explicitly told how to. This technique enables machines to solve problems that are either too complicated for traditional algorithms to handle or lack a working traditional algorithm altogether.
\subsection{The Internet of Things}
The Internet of Things is a relatively recent conceptual technology which originated in the mid 1980's and 1990's with network-enabled appliances and smart devices \cite{weiser91}. Since then, many advancements have been made in the fields of embedded systems, microelectronics, and real-time analytics. While the IoT is still not a fully fledged technology ready for implementation in all scenarios, these advancements have increased its viability tremendously since the idea's inception. Similar to machine learning, the IoT comes in many different forms and interpretations. The implementations most useful for sustainable design are wireless sensor networks and automation networks \cite{atzori10}. 

\clearpage
\printbibliography
\end{document}
