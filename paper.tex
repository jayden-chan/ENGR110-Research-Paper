% Notes:
% - Remove all instances of \blindtext
% - Remove all instances of [placeholder]
% - Check that all quotes and mentions of authors are cited
% - Add types of ML to ML info section
% - Add smart system outline to first par of section 3
% - Don't throw uncontrollably

\documentclass[letterpaper]{article}
\usepackage{blindtext}
\usepackage{fancyhdr}
\usepackage{indentfirst}

\setlength{\parindent}{5ex}

\usepackage[
    backend=biber,
    style=ieee,
    url=true,
    citestyle=numeric-comp
]{biblatex}

\addbibresource{references.bib}

\pagestyle{fancy}
\fancypagestyle{customPage}{
    \fancyhf{}
    \renewcommand{\headrulewidth}{0pt}
    \fancyhead[R]{\thepage}
    \fancyhead[C]{\textit{Machine Learning, The Internet of Things, and Sustainable Design}}
}

\begin{document}
\begin{titlepage}
    \centering
    {\scshape\huge UNIVERSITY OF VICTORIA \par}
    \vspace{1cm}
    {\scshape\LARGE Faculty of Engineering \par\Large ENGR 110 A04\par}
    \vspace{1cm}
    {\scshape\Large Research Paper \par}
    \vspace{1.5cm}
    {\huge\bfseries \emph{Machine Learning, The Internet of Things, and Sustainable Design} \par}
    \vspace{2cm}
    {\Large\itshape Jayden Chan \par}
    \vfill
    {\large November 30, 2017\par}
\end{titlepage}

\pagestyle{customPage}

\begin{abstract}
    The abstract is written last; this is placeholder text for formatting purposes. Lorem ipsum dolor sit amet, consectetur adipiscing elit. Nam ut leo venenatis, malesuada elit in, efficitur massa. In sodales egestas egestas. Nulla et orci in enim aliquam pretium. Nam auctor vestibulum ipsum rutrum ullamcorper. Sed porta in mauris consequat vehicula. Etiam bibendum a urna nec tincidunt. Curabitur justo turpis, faucibus ut malesuada a, luctus id nisi. Integer mauris nulla, lacinia nec nisl vel, posuere laoreet nulla. Nunc semper diam nec odio pretium, in volutpat ante placerat.
\end{abstract}

\section{Introduction} \label{intro}
The recent surge in computational power and data set availability has led to a significant increase in the use of artificial intelligence and machine learning (ML) algorithms \cite{chan17}. In addition to this, advances in microelectronics and network infrastructure have enabled hardware manufacturers to create internet-connected smart devices for use in homes and businesses. This technology is commonly referred to as the Internet of Things (IoT). The combination of these two systems can aid significantly in the development of sustainable homes and businesses. Network-enabled cyber-physical devices capable of gathering information from their surroundings can provide optimization algorithms with the data they need to improve automation, energy use efficiency, and [placeholder]. However, integrating these technologies is not without its challenges; the most prominent which involve ethical concerns or technical barriers \cite{perisic16, mccalman17, vlacheas13}. Despite these challenges, the strengths of ML and the IoT strongly outweigh their shortcomings, making them a viable future technology for designing sustainable homes and businesses.

\section{The Technologies} \label{info}
Before discussing the viability of machine learning and the IoT, it is important to understand how they work. The combination of machine learning and the IoT as a single technology will be referred to as a ``smart system''.

\subsection{Machine Learning} \label{MLinfo}
While the concept of artificial intelligence predates the common era \cite{mccorduck04}, the term ``machine learning'' was not coined until 1959 by Arthur Samuel \cite{samuel59}. The basic principle behind machine learning is to allow computers to complete tasks without being explicitly told how to. This technique enables machines to solve problems that are either too complicated for traditional algorithms to handle or lack a working traditional algorithm altogether. There are a variety of modern machine learning implementations in use today, and they each [placeholder]

\subsection{The Internet of Things} \label{IoTinfo}
The Internet of Things is a relatively modern conceptual technology which originated in the mid 1980's and 1990's with network-enabled appliances and smart devices \cite{weiser91}. Since then, many advancements have been made in the fields of embedded systems, microelectronics, and real-time analytics. While the IoT is still not a fully fledged technology ready for implementation in all scenarios, these advancements have increased its viability tremendously since the idea's inception. Similar to machine learning, the IoT comes in many different forms and interpretations. The implementations most useful for sustainable design are wireless sensor networks and automation networks \cite{atzori10}. These, in tandem with ML optimization algorithms, can serve to aid greatly in the development of sustainable designs.

\section{ML and the IoT for Sustainable Designs} \label{main}
A smart system is composed of three components: a wireless sensor network, an actuator network, and a machine learning algorithm. The primary task accomplished by a smart system in a home or business is to increase energy use efficiency by means of automation. This task is conducted in the following manner, based on implementations by \textcite{pang15} and \textcite{risteska17}:

\begin{center}
\begin{enumerate}
\item
    The sensor and actuator networks are installed in the building. The greater the coverage of the network, the better the results.
\item
    Over time, information is collected by the sensor network. Again: the more data, the better the results. Typical metrics to measure are active occupancy hours, energy usage rates, transportation patterns (this extends beyond the building itself but useful in many respects), and [placeholder].
\item
    The data is anonymously send to a cloud computing service where it has a machine learning algorithm applied to it. The algorithm outputs an `optimization plan', which is a series of automated tasks that the actuator network can perform in order to reduce energy usage.
\item
    The optimization plan is sent back to the building where the actuator network applies it on a daily basis. An actuator network is a set of hardware controllers capable of manipulating their surroundings. The optimization plan can include actions such as reducing the climate control in unused rooms, automatically turning off lights when they are unused, or load-balancing a multi-appliance set to ensure they are all operating at peak efficiency.
\item
    Steps 2-4 are repeated in a continuous cycle. The ML algorithm improves with more data, and a setup which constantly supplies it with new data allows it to adapt to lifestyle changes of home or business owners.
\end{enumerate}
\end{center}

The system outlined above has significant advantages when compared to traditional `smart building' implementations. First, the system benefits from the ability to adapt to changing conditions. A significant setback of traditional automated systems is that if the operating conditions change in a way that affects their performance, they must be manually re-programmed. This system eliminates this problem by constantly adapting to and improving based on the operating conditions. Another strength of this implementation inherent to its design is the interconnectedness of its components. While a traditional automation system may be able to superficially improve efficiency in select areas of a building, an interconnected system can monitor the entire building as a single ecosystem and allow \emph{all} of the components to work together and improve efficiency.

\section{Refutation} \label{refute}
As mentioned in section \ref{intro}, machine learning and the IoT are relatively modern and untested technologies. As such, they are not without their flaws and it is expected that critics would be skeptical of their performance. \par

Since the rise of the information era, privacy has been a large concern among internet users. A significant issue raised by the general public and industry professionals is the moral and ethical implications of collecting such specific information about the homes and workplaces of customers \cite{perisic16, mccalman17}. There are two foreseeable solutions to this problem: the first is to make the software controlling the system open-source. Open-source software is software who's design is publicly available, meaning anyone can view and contribute to it \cite{opensource}. This way, customers and industry professionals can examine the code to ensure that the collected data is not being used in a malicious manner. The second solution is to not have the data sent to a cloud computing service in the first place. This way, the data is contained within the building's infrastructure and all of the algorithms are processed in-house. No data is sent off site, thereby eliminating any privacy concerns. \par

This section is unfinished and still requires some additional writing.

\section{Conclusion} \label{conc}
As we have seen, machine learning and the Internet of Things are extremely powerful technologies for improving the sustainability of homes and businesses. Although smart systems have not yet reached the mainstream market, they are already in use in many commercial and industrial applications, and it is looking as though they have a bright future ahead of them. As the foundational technologies for these systems improve and the solutions to their challenges are refined, smart systems will become an essential piece of technology as ubiquitous as the telephone. 

\clearpage
\printbibliography
\end{document}
